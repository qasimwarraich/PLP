\documentclass[conference, a4paper]{IEEEtran}
% \IEEEoverridecommandlockouts
% The preceding line is only needed to identify funding in the first footnote. If that is unneeded, please comment it out.

\usepackage{cite}
\usepackage{amsmath,amssymb,amsfonts}
\usepackage{algorithmic}
\usepackage{graphicx}
\usepackage{textcomp}
\usepackage{xcolor}
\usepackage{hyperref}
\usepackage[T1]{fontenc}
\usepackage{listings}

% Minted config
\definecolor{bg}{HTML}{282828} %gruvbox bg col
\usepackage{minted}
\setminted[cpp]{
baselinestretch=1.2,
bgcolor=bg,
fontsize=\footnotesize,
}
\usemintedstyle{gruvbox}
\renewcommand\listingscaption{Code}
%

\def\BibTeX{{\rm B\kern-.05em{\sc i\kern-.025em b}\kern-.08em
    T\kern-.1667em\lower.7ex\hbox{E}\kern-.125emX}}
\begin{document}

\title{Programming Languages and Paradigms: C++\\
{\footnotesize \textsuperscript{*}Note: Sub-titles are not captured in Xplore and
should not be used}
}

\author{\IEEEauthorblockN{Qasim Warraich}
\IEEEauthorblockA{\textit{Department of Informatics} \\
\textit{University of Zurich}\\
Zurich, Switzerland\\
qasim.warraich@uzh.ch}
}

\maketitle

\begin{abstract}
This is a semester paper for the Programming Languages and Paradigms seminar at the University of Zurich under the direction of Professor Carol Alexandru. This paper aims to provide an concise overview about the history of the C++ programming language, some of the features that make it unique and offer general overview and reflection on the experience programming in C++ during the semester.\\
\end{abstract}

\begin{IEEEkeywords}
cpp, c++, programming languages
\end{IEEEkeywords}

\section{Introduction}
The C++ programming language is an extremely successful and widely used multi paradigm general purpose programming language. It has many reputations, some admire it for it's versatility, some adore it's portability and low level features and others criticise it for it's lack of direction and complexity but it has managed to stand the test of time for almost four decades and shows no signs of slowing down. It is still one of the most widely used programming languages and finds it's self at the core of many popular pieces of software e.g. Mozilla Firefox, MySQL, certain Adobe software, Microsoft Office, KDE, etc. A non exhaustive list of C++ applications can be found on the creator of the language Bjarne Stroustrup's website here \cite{cppapplications} .

\section{History}
The history of the C++ programming language, or rather the pre-history as it was then known as \textit{C with Classes}, begins in 1979 when Bjarne Stroustrup was an employee at AT\&T Bell Labs. As the name suggests the motivation behind the creation of the language was to try to synthesize the complementary characteristics of the C programming language, also a Bell Labs creation, and Obejct Orientated languages. Bjarne found Simula67's Object Orientation style to be helpful for large scale software development during his experience with the language in his PHD Thesis, but was unimpressed by it's performance characteristics. The goal of C with Classes was build upon C, which is renowned for it's high performance and to expand it with Simula like Object Orientation \cite{cppevolving}.

The initial implementation of C with Classes added features on top of an existing C compiler basically creating a C with Classes preprocesseor called Cpre which processed the code in to C which was then compiled into a binary. This version of the implementation included features such as; classes, derived classes, public/private access, constructors and destructors, call and return functions, friend classes, strong type checking and conversion, default and inline functions and overloading of the assignment operator. 

In 1982 Stroustrup started work on the successor to C with classes which was where the name C++ was born. The name is a play on the increment operator "++" in C, the actual creation of the name is credited to Rick Miscitti \cite{cpphistory}. The name reflected Bjarne's desire to not only create a successor to C with classes but also to C. This ambition can still be found on the page on C++ on Bjarne Stroustrup personal website where he claims C++ "is a better C" \cite{cpphome}. Along with the new name came some new features. Some of the new features of C++ at this early stage included; virtual functions, function name and operator overloading, references, constants, type-safe free-store memory allocation (new/delete), improved type checking, and BCPL style single-line comments with two forward slashes (//) \cite{cpphistory}. 

By 1984 the Cfront compiler for C++ was implemented by Bjarne, originally written in C with classes but then written in C84, the very first "standard" for C++. Also in 1984, Bjarne implemented the stream input/output \textbf{(IO)} library which employed the idea of IO being an operator rather than a function. This idea has been accredited to Doug Mcllroy of Unix pipes fame \cite{cpphistory}. 

1985 brought with it the release of the first edition of \textit{The C++ Programming Language} \cite{cpp_1985} which became the definitive reference for the language as this predates any actual standards for C++. The first commercial version of C++ became available in October of 1985 \cite{cppinvention}.

In 1989 C++ 2.0 was released alongside the updated version of the official reference book \textit{The C++ Programming Language} coming in 1991. Some of the new features in C++ 2.0 included multiple inheritance, abstract classes, static member functions, const member functions, and protected members. In 1990, an annotated reference manual was published and became the basis for the future standardisation of C++. Later features introduced to C++ 2.0 included some of the most prolific of the language such as; templates, exceptions, namespaces, new casts and a Boolean type. 1993, marked the beginning of Alex Stepanov's work on the Standard Template Library \textbf{(STL)}. The STL is now essentially as much a part of what we now call the C++ standard as the language itself. 

1998, marked the first standardised release of C++, C++98. This brought with it the standard naming convention we are familiar with today ("C++" + 2 digit year"). The standard of C++ would now be defined by an ISO/IEC committee which would work together on shaping the languages future features and implementation. After this point subsequent standards brought on relatively minor changes up until the release of C++11 in 2011. C++11 was a major update bringing with it heaps of new features to the language and greatly expanding the STL. Some of the most significant new additions included; Variadic templates, Hash tables, Range based for loops, auto as a type inference operator rather than for storage duration, a nullptr type, lambdas and smart pointers. C++11 began the era we can term as "modern C++". Subsequent standards included mostly minor additions bringing us to where we are today as of the righting of this paper, C++20.


\subsection{Relavent Literature}
Relevant literature can be found at \cite{cpphome,cpp_1985,cpphistory,cppoverview,cppevolving,alexandrescu2001modern}

\section{The C++ Programming Language}

\begin{listing}[h]

\begin{minted}{cpp}

#include <iostream>

int main()
{
    std::cout << "Hello World\n";
    return 0;
}

\end{minted}

\caption{Classic Hello World Example} \label{listing:1}
\end{listing}
The C++ programming language comprises of two main implementation goals. It aims to provide direct access to hardware level features, primarily through it's C subset and then build on those mappings by providing abstractions. During a lecture at the University of Edinburg entitled \textit{The essence of C++} Bjarne Stroustrup described the language as "Stroustrup describes C++ as "a light-weight abstraction programming language [designed] for building and using efficient and elegant abstractions" \cite{EssenceofC++}. 

\subsection{ The C in C++}

The syntax of C++ is mostly inherited from C as illustrated by the Hello World code in \hyperref[listing:1]{Code 1}. This trivial example highlights the clear similarities between the two languages but also some ways in which they diverge. This is clearly seen by C++'s usage of an operator (<< and >>) to handle IO rather than the C equivalent library functions. This extremely simple example also introduces two more features that are extremely characteristic of CPP. Firstly the usage of namespaces as shows by the reference to "std" followed by the scope operator (::) and secondly the usage of of the STL where the function "cout" resides. The similarities and compatibility with C is a reoccurring theme in the design of C++ and a note worthy example of this is the fact that C++ has full support for both C styled null terminated strings and it's very on string class. In fact command line argument parsing is exactly as it is in C and uses C style strings. Similar to C, C++ also employs the use of a main function as the starting point of the program. Compatability with C is not 100\% and C is by no means a subset of C++, but it's influence on the language and the design around keeping some semblance of compatibility (even of libraries) are a clear indicator of the roots of the language. A more comprehensive review can be found here \cite{calder1994quantifying}.

\subsection{Memory Management}

C++ is well known for it's deliberate design choice to not support garbage collection natively. The language does however support four styles of memory management. 

\begin{itemize}
    \item Static Storage: Objects created before the entry of main() and destroyed in reverse order after main() exits.
    \item Thread Storage:  Objects are created similarly as in static storage but the creation time is thread dependent. Creation is just prior to thread creation and destruction occurs just after the thread has been joined. 
    \item Automatic Storage: This is scope based memory management and creation occurs at the point the execution passes the declaration and destruction follows as son as the local scope of declaration is closed. Member variables are created along side the parent and are destroyed when the parent is. This sort of automatic management is the source of a widely used paradigm in C++ termed \textit{Resource Acquisition Is Initialization} or \textbf{(RAII)}.
    \item Dynamic Storage: In dynamic storage objects have a dynamic life span and creation/deletion is explicitly handled by the respective operators new/delete. As of C++11 and the introduction of smart pointers the usage of new has been advised against in favour of smart pointers such as make\_unique<T> and make\_shared<T>.

\end{itemize}

\subsection{Templates}

C++ uses a templates to enable generic programming. Template are compile time parametrised functions or classes written without specific instantiation arguments. Instantiation is handled at compile time and uses a substitution approach. Parameterization can be done by types, compile time constants and other templates. An example of a generic quicksort lomuto partitioning using templates done during the seminar can be found here \hyperref[listing:2]{Code 2} and here \href{https://github.com/qasimwarraich/PLP/blob/master/A2/Task1/quicksort/quicksort.cpp}{(Github)}.


\begin{listing}[h]
\begin{minted}{cpp}
template <class T>
int partition(std::vector <T> &arr, int low, int high)
{
    auto pivot = arr[low];
    int i = low;

    for (int j = i + 1; j <= high; j++) {
        if (arr[j] <= pivot){
            i++;
            std::swap(arr[i], arr[j]);
        }
    }
    std::swap(arr[i], arr[low]);
    return i; 
}
        
    \end{minted}
    
    \caption{A generic implementation of lomuto partitioning using templates}
\label{listing:2}
\end{listing}

\subsection{The Standard Template Library}


\begin{listing}[h]
\begin{minted}{cpp}

//auto is also useful for reducing the verbosity of
//the code. For instance, instead of writing

for (std::vector<int>::const_iterator itr = myvec.cbegin();
    itr != myvec.cend();
    ++itr)

//the programmer can use the shorter

for (auto itr = myvec.cbegin();
    itr != myvec.cend();
    ++itr)

//which can be further compacted since "myvec"
//implements begin/end iterators:

for (auto& x : myvec)
        
    \end{minted}

    \caption{Modern C++ example: auto operator and range based for loops \cite{C++11_2021}}
\label{listing:3}
\end{listing}









\section{Discussion}

\section{Conclusion}

\clearpage

\bibliographystyle{IEEEtran}
\bibliography{cpp}

\end{document}

