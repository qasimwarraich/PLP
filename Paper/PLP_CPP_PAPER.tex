\documentclass[conference, a4paper]{IEEEtran}
% \IEEEoverridecommandlockouts
% The preceding line is only needed to identify funding in the first footnote. If that is unneeded, please comment it out.

\usepackage{cite}
\usepackage{amsmath,amssymb,amsfonts}
\usepackage{algorithmic}
\usepackage{graphicx}
\usepackage{textcomp}
\usepackage{xcolor}
\usepackage{hyperref}
\usepackage[T1]{fontenc}
\usepackage{listings}

% Minted config
\definecolor{bg}{HTML}{282828} %gruvbox bg col
\usepackage{minted}
\setminted[cpp]{
baselinestretch=1.2,
bgcolor=bg,
fontsize=\footnotesize,
linenos
}
\usemintedstyle{gruvbox}
\renewcommand\listingscaption{Code}

%

\def\BibTeX{{\rm B\kern-.05em{\sc i\kern-.025em b}\kern-.08em
    T\kern-.1667em\lower.7ex\hbox{E}\kern-.125emX}}
\begin{document}

\title{Programming Languages and Paradigms: C++\\
{\footnotesize \textsuperscript{*}Note: Sub-titles are not captured in Xplore and
should not be used}
}

\author{\IEEEauthorblockN{Qasim Warraich}
\IEEEauthorblockA{\textit{Department of Informatics} \\
\textit{University of Zurich}\\
Zurich, Switzerland\\
qasim.warraich@uzh.ch}
}


\maketitle

\begin{abstract}
This is a semester paper for the Programming Languages and Paradigms seminar at the University of Zurich under the direction of Professor Carol Alexandru. This paper aims to provide an concise overview about the history of the C++ programming language, some of the features that make it unique and offer general overview and reflection on the experiance programming in C++ during the semester.\\
\end{abstract}

\begin{IEEEkeywords}
cpp, c++, programming languages
\end{IEEEkeywords}

\section{Introduction}
The C++ programming language is an extremely successfull and widely used multi paradigm general purpose programming language. It has many reputations, some admire it for it's versitility, some adore it's portability and low level features and others crticise it for it's lack of direction and complexity but it has managed to stand the test of time for almost four decades and shows no signs of slowing down. It is still one of the most widely used programming languages and finds it's self at the core of many popular pieces of software e.g. Mozilla Firefox, MySQL, certain Adobe software, Microsoft Office, KDE, etc. A non exhaustive list of C++ applications can be found on the creator of the language Bjarne Stroustrup's website here \cite{cppapplications} .

\begin{listing}[h]
    
\begin{minted}{cpp}


#include <iostream>

int main()
{
    std::cout << "Hello World" << '\n';
    return 0;
}
\end{minted}
\caption{Hello World Example}
\label{listing:1}
\end{listing}

\section{History}

The history of the C++ programming language, or rather the pre-history as it was then known as \textit{C with Classes}, begins in 1979 when Bjarne Stroustrup was an employee at AT\&T Bell Labs. As the name suggests the motivation behind the creation of the language was to try to syntesize the complementatry charaterestics of the C programming language, also a Bell Labs creation, and Obejct Orientated languages. Bjarne found Simula67's Object Orientation style to be helpful for large scale software development during his experiance with the language in his PHD Thesis, but was unimpressed by it's performance charateristics. The goal of C with Classes was build upon C, which was known for it's high performance and to expand it with Simula like Object Orientation.

\subsection{Relavent Literature}
Relevant material can be found at \cite{cpphome,cpp_1986,cpphistory,cppoverview,cppevolving,alexandrescu2001modern}

\section{Cpp}

\section{Discussion}

\section{Conclusion}

\bibliographystyle{IEEEtran}
\bibliography{cpp}

\end{document}

